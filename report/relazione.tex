\documentclass[a4paper, 12pt]{article}
\usepackage{fullpage}
\usepackage[italian]{babel}
\usepackage[utf8]{inputenc}
\usepackage{hyperref}
%\usepackage[round]{natbib}
\usepackage{amsmath}
\usepackage{graphicx}
\usepackage{float}
\frenchspacing
\newcommand{\strong}[1]{ \textbf{#1}}

\title{Programmazione Concorrente e Distribuita \\ A.A. 2014/2015 \\
\vspace{2 mm}{\small PuzzleSolver - Parte 3}}
\author{Antonio Cavestro \\ \vspace{2 mm} {\small Matricola: 1050878}}
\date{}

\begin{document}

\maketitle

\section{Abstract}

La terza parte del progetto di Programmazione Concorrente e Distribuita
prevede la divisione del risultato della seconda parte - concorrenza - in due
programmi: il \emph{client}, che deve leggere il file di input e spedire i pezzi
del puzzle al secondo programma, il \emph{server}, che invece deve risolvere
il problema e rispedire al client la soluzione.

Nella seguente relazione, allegata al progetto, verranno introdotte e discusse
tutte le scelte progettuali compiute per lo sviluppo di questa terza parte.

\subsection{Note su sviluppo e avvio}
\subsubsection{Script di avvio}
Sono presenti i seguenti script:
\begin{itemize}
\item \verb|startrmiregistry.sh|: avvia il registro RMI;
\item \verb|puzzlesolverclient.sh|: avvia il client, da invocare come da
specifica;
\item \verb|puzzlesolverserver.sh|: avvia il registro RMI e il server, da
invocare come da specifica;
\item \verb|puzzlesolverserverwr.sh|: avvia il server senza avviare il registro
RMI, anche quest'ultimo da invocare come da specifica.
\end{itemize}
\subsubsection{Cartelle rilevanti}
\begin{itemize}
\item \verb|src|: contiene i file sorgenti;
\item \verb|build|,: viene creata durante la compilazione, contiene i file
compilati.
\end{itemize}
\subsection{Ambiente di sviluppo}
Il software, oltre ad essere testato sulle macchine del laboratorio presso il
plesso Paolotti, dotate di Java 7, è stato sviluppato e testato anche nel
seguente ambiente:
\begin{itemize}
\item \textbf{Sistema operativo:} Mac OS X v 10.10.2;
\item \textbf{IDE:} Eclipse 4.4.1;
\item \textbf{Java:} 1.8.0\_25.
\end{itemize}

\section{Introduzione all'algoritmo distribuito}
\subsection{Parsing del file di input}
Il parsing del file di input è compito della classe \verb|TileParser|, già
esistente nelle precedenti parti del progetto. Viene costruita con la stringa
contenente il percorso del file e restituisce una struttura dati di tipo
\verb|ConcurrentHashMap|. Quest'ultima viene poi spedita al server
per la risoluzione del puzzle.
\subsection{Risoluzione del puzzle}
Il server esegue la risoluzione del puzzle tramite l'oggetto
\verb|PuzzleBuilder|, costruito nelle parti precedenti e ottimizzato per trovare
la soluzione in maniera concorrente. Una volta terminato, la risposta viene
spedita al client, incapsulando informazioni sul processo risolutivo,
l'eventuale soluzione o l'eventuale errore (nella sezione \ref{server-response}
verrà approfondita questa parte).
\subsection{Output}
Il client riceve la risposta con la quale costruisce il file di output oppure
visualizza un messaggio di errore.

\section{Cambiamenti a codice e gerarchia}
\subsection{PuzzleSolverClient}
Questa classe ricalca nelle funzionalità ciò che svolgeva la classe
\verb|PuzzleSolver| nelle precedenti parti del progetto. Essa contiene un metodo
\verb|main| che implementa la parte client dell'algoritmo distribuito, ossia
tenta di:
\begin{enumerate}
\item leggere il file di input e di estrarne i pezzi tramite \emph{parsing};
\item ottenere dal server l'oggetto risolutore remoto;
\item richiedere al suddetto la risoluzione del puzzle, spedendogli i pezzi
estratti;
\item ottenere la risposta, salvare su output o restituire un messaggio di
errore.
\end{enumerate}
\subsection{PuzzleSolverServer}
È il \verb|main| principale del programma lato server. Il suo unico ma
fondamentale scopo è collegare l'oggetto risolutore al registro RMI, perché sia
raggiungibile da ogni client interessato.
\subsection{Package netutils}
Questo package contiene tutte le interfacce e gli oggetti interessati
alla distribuzione dell'applicazione.
\subsubsection{Solver}
Interfaccia che estende \strong{Remote}, è il tipo di oggetto che il client si
aspetta come risolutore remoto. Contiene il seguente metodo:
\begin{itemize}
\item \verb|public SolverResponse solvePuzzle(ConcurrentHashMap<String, Tile>)|:
\\ definisce la risoluzione di un puzzle e richiede come parametro la
struttura dati che contiene i vari pezzi. Ritorna un oggetto che incapsula la
risposta del server e può lanciare eccezioni del tipo \verb|RemoteException|.
\end{itemize}
\subsubsection{RemoteSolver}
Implementa l'interfaccia \verb|Solver| ed estende la classe
\verb|UnicastRemoteObject|. Contiene il seguente metodo:
\begin{itemize}
\item \verb|public SolverResponse solvePuzzle(ConcurrentHashMap<String, Tile>)|:
\\ è l'\emph{override} del metodo dell'interfaccia \verb|Solver|. Crea un
oggetto \verb|PuzzleBuilder| a cui affida la struttura dati con i vari pezzi e
ne richiede la soluzione. Infine, impacchetta l'oggetto di risposta di tipo
\verb|SolverResponse| e lo restituisce.
\end{itemize}
\subsubsection{SolutionStatus}
Si tratta di un \verb|enum| che indica i possibili esiti di un processo di
risoluzione di un puzzle. Essi sono:
\begin{itemize}
\item \verb|SOLVED|: puzzle risolto con successo;
\item \verb|ERROR|: puzzle non risolto.
\end{itemize}
Fa parte delle informazioni incapsulate da oggetti di tipo \verb|SolverResponse|
e serve al client per capire se può lavorare all'output del puzzle o gestire
degli errori.
\subsubsection{SolverResponse}
Interfaccia che definisce la struttura di una risposta del server. Estende
\strong{Serializable} e definisce i seguenti metodi:
\begin{itemize}
\item \verb|public SolutionStatus getStatus()|: ottiene l'oggetto di tipo
\verb|SolutionStatus| che contiene lo stato della risoluzione;
\item \verb|public Puzzle getSolution()|: ottiene la soluzione del Puzzle (se
esiste);
\item \verb|public Exception getError()|: ottiene errori di risoluzione (se ci
sono stati).
\end{itemize}
\subsubsection{RemoteSolverResponse}
Classe che implementa l'interfaccia \verb|SolverResponse| e ne fa
l'\emph{ovveride} dei metodi. Si limita semplicemente a raccogliere, tramite il
suo costrutture, tre oggetti:
\begin{itemize}
\item un oggetto \verb|SolutionStatus| che contiene lo stato della risoluzione;
\item un oggetto \verb|Puzzle| che contiene l'eventuale soluzione;
\item un oggetto \verb|Exception| che contiene l'eventuale errore.
\end{itemize}

\section{Comunicazione client-server}
Il protocollo di comunicazione client-server si basa su:
\begin{itemize}
\item un oggetto collegato al registro RMI che permette al client di avviare una
risoluzione;
\item una struttura dati contenente i pezzi dal puzzle, da spedire al server
per mezzo dell'oggetto elencato precedentemente;
\item un oggetto di risposta che il server, dopo aver salvato al suo interno
dati e informazioni sul lavoro svolto, spedisce al client.
\end{itemize}

Nelle successive sottosezioni verranno analizzati i passaggi che portano allo
scambio di questi oggetti attraverso la rete.

\subsection{Oggetti nel registro RMI}
Al registro RMI viene collegata un'istanza della classe \verb|RemoteSolver|, che
viene recuperata dal client con il tipo dell'interfaccia, \verb|Solver|.
Quest'ultima implementa \verb|Remote|, per cui al richiedente viene fornito un
riferimento remoto, ossia uno \emph{stub}. La classe \verb|RemoteSolver| non
contiene campi dati, per cui non viene trasferito alcun altro oggetto.
\subsection{Richiesta di risoluzione}
La richiesta di risoluzione da parte di un client, attraverso la chiamata del
metodo \verb|solvePuzzle| dell'oggetto \verb|Solver|, comporta la creazione
nella JVM remota del server di un \emph{thread} che esegue il codice del
metodo chiamato.

A \verb|solvePuzzle| viene passata come parametro una \verb|ConcurrentHashMap|,
la quale classe implementa l'interfaccia \verb|Serializable|. Stessa cosa per
\verb|Tile|, interfaccia di oggetti di cui la struttura dati è contenitrice. Per
cui, al server viene mandata un'intera copia dell'insieme dei pezzi.

In questo modo il metodo remoto non viene rallentato da continue richieste di
accesso ai dati originali contenuti nella JVM del client.
\subsection{Risposta}
\label{server-response}
Il metodo remoto \verb|solvePuzzle| invocato dal client tenta di risolvere il
puzzle tramite un oggetto \verb|PuzzleBuilder|. A seconda se quest'ultimo
restituisce un \verb|Puzzle| o lancia un'eccezione, il metodo costruisce la
risposta \verb|SolverResponse|:
\begin{enumerate}
\item se la risoluzione va a buon fine, ossia la chiamata al metodo
\verb|solvePuzzle| di \verb|PuzzleBuilder| torna un oggetto \verb|Puzzle|, la
risposta viene costruita impostando a \verb|SOLVED| lo stato e a \verb|NULL| il
campo errore, allegando inoltre la soluzione;
\item se la chiamata fallisce in quanto viene lanciata un'eccezione, la risposta
ha come stato \verb|ERROR|, la soluzione è pari a \verb|NULL| e il campo
errore contiene la stessa eccezione lanciata.
\end{enumerate}

Da notare che al client torna in ogni caso una copia esatta della risposta
formulata dal server, in quanto \verb|SolverResponse| estende l'interfaccia
\verb|Serializable| e contiene tutti oggetti a loro volta serializzabili:
\begin{itemize}
\item gli attributi \verb|enum| ed \verb|Exception| lo sono per specifica del
linguaggio;
\item l'attributo \verb|Puzzle| estende a sua volta l'interfaccia
\verb|Serializable| e così fanno a sua volta i suoi campi dati (\verb|Tile|
compreso).
\end{itemize}
\subsection{Eventuali errori di risoluzione}
Una volta che il client ha ricevuto la risposta, controlla il campo riguardante
lo stato: se il suo valore è \verb|SOLVED| allora può stampare l'output
utilizzando l'oggetto \verb|Puzzle| del campo soluzione. Altrimenti, se lo stato
corrisponde a \verb|ERROR|, stampa a video che la risoluzione non è andata a
buon fine allegando le informazioni contenute nell'eccezione del campo errore
della risposta.

Si noti inoltre che, per via della robustezza del client (sezione
\ref{robust-client}), esso sa per certo che la soluzione è ben definita e
mai \verb|NULL|. Questo perché può esserlo soltanto nel caso in cui il server
smetta di rispondere (ad esempio per via di un \emph{crash}), cosa che non può
succedere in quanto si tratta di un caso gestito tramite ascolto delle eccezioni
(es \verb|RemoteException|) durante la ricezione della risposta.

\section{Robustezza del software}
\subsection{Client}
\label{robust-client}
Il client necessita della rete in due occasioni: il \emph{lookup} dell'oggetto
remoto di tipo \verb|Solver| e la richiesta di risoluzione del puzzle. In
entrambi i casi, vengono gestite tutte le eccezioni eventualmente lanciate dalle
chiamate remote, bloccando il programma per evitare che prosegua in stati
inconsistenti, ossia che:
\begin{itemize}
\item il riferimento all'oggetto \verb|Solver| sia vuoto (\verb|NULL|),
provocando un errore al momento della richiesta di risoluzione;
\item il riferimento alla soluzione \verb|SolverResponse| sia vuoto, provocando
un errore al momento dell'analisi della risposta.
\end{itemize}

Il blocco del programma in tutte le possibilità in cui si verifica uno stato
inconsistente è stato scelto per mantenere il codice semplice e facilmente
manutenibile. Così facendo s'introducono postcondizioni all'uscita dalle
chiamate remote che facilitano il lavoro del programmatore, a patto di essere
ben documentate.
\subsection{Server}
Il server si collega alla rete nel momento in cui condivide il suo oggetto
\verb|Solver|. Successivamente, rimane in attesa di una chiamata da qualche
client, la quale esegue il metodo richiesto su un \emph{thread} separato.

La condivisione dell'oggetto remoto da parte del server è protetta, come nel
client, da un blocco \emph{try-catch} che impedisce l'avanzamento del
programma in caso di fallimento del \emph{bind} di \verb|Solver|.

Inoltre, nel caso di un \emph{crash} da parte di un client, il server non
subisce alcun malfunzionamento. Il metodo \verb|solvePuzzle| di \verb|Solver| è
\emph{idempotente}, ossia può essere chiamato ripetutamente senza causare
interferenze o incoerenze nei cambi di stato.
\subsection{Registro RMI}
Nel caso in cui il registro RMI non fosse raggiungibile dalla rete, sia il
client che il server non possono accedervi e visualizzano a schermo un
relativo messaggio di errore. Non è previsto che il server debba
\emph{accorgersi} di una irraggiunbilità del registro finché si trova in
esecuzione.
\section{Note sulla robustezza}
Esistono metodologie di gestione dei fallimenti di rete che non sono state
implementate nel progetto per mancanza di tempo e risorse. In particolare,
esistono eccezioni che consentono di tentare il recupero della connessione.

Ad esempio, quando il client tenta il recupero dell'oggetto remoto
\verb|Solver| potrebbe, in caso di fallimento e conseguente lancio
dell'eccezione \verb|RemoteException|, mettersi in attesa per un tempo fissato e
poi riprovare un numero finito di volte prima di \emph{rinunciare}.

Per evitare ripetizioni del codice ogni volta in cui ci si trova in presenza di
una chiamata a un metodo remoto, si potrebbe implementare un pattern
\emph{strategy} a cui passare il controllo delle suddette chiamate.
Quest'approccio favorisce l'estendibilità del codice in quanto permette di
cambiare nel tempo le modalità di gestione della connesione di rete.
\end{document}