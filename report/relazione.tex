\documentclass[a4paper, 12pt]{article}
\usepackage[italian]{babel}
\usepackage[utf8]{inputenc}
\usepackage[round]{natbib}
\usepackage{amsmath}
\usepackage{graphicx}
\usepackage{float}
\frenchspacing
\newcommand{\strong}[1]{ \textbf{#1}}

\title{Programmazione Concorrente e Distribuita \\ A.A. 2014/2015 \\
\vspace{2 mm}{\small PuzzleSolver - Parte 3}}
\author{Antonio Cavestro \\ \vspace{2 mm} {\small Matricola: 1050878}}
\date{}

\begin{document}

\maketitle

\section{Abstract}

La terza parte del progetto di Programmazione Concorrente e Distribuita
prevede la divisione del risultato della seconda parte - concorrenza - in due
programmi: il \emph{client}, che deve leggere il file di input e spedire i pezzi
del puzzle al secondo programma, il \emph{server}, che invece deve risolvere
il problema e rispedire al client la soluzione.

Nella seguente relazione, allegata al progetto, verranno introdotte e discusse
tutte le scelte progettuali compiute per lo sviluppo di questa terza parte.

%\section{Introduzione all'algoritmo distribuito}
%\subsection{Parsing del file di input}
%\subsection{Risoluzione del puzzle}
%\subsection{Output}

%\section{Cambiamenti a codice e gerarchia}

%\section{Comunicazione client-server}
%\subsection{Oggetti nel registro RMI}
%\subsection{Richiesta di risoluzione}
%\subsection{Risposta}
%\subsection{Eventuali errori di risoluzione}

%\section{Robustezza del software}
%\subsection{Client}
%\subsection{Server}
%\subsection{Registro RMI}

%\section{Altre modifiche}

\end{document}